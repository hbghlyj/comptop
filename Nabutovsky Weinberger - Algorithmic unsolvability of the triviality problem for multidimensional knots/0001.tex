
Abstract. We prove that for any fixed $n \geq 3$ there is no algorithm deciding whether or not a given knot $f: S^{n} \rightarrow \mathbb{R}^{n+2}$ is trivial. Some related results are also presented.

The classical result of W. Haken ([H]) is that there exists an algorithm deciding whether or not a given knot in $\mathbb{R}^{3}$ is trivial. Our main result (Theorem 1 below) establishes that this result is not true for multidimensional knots. Our proof is in the same spirit as the algorithmic unsolvability of the homeomorphism problem for manifolds of dimension $\geq 4$, but there is still a subtlety that prevents us from dealing with knots in 4 -space as the discerning reader will see.

THEOREM 1. For any fixed $n \geq 3$ there is no algorithm deciding whether or not a given knot $f: S^{n} \rightarrow \mathbb{R}^{n+2}$ is trivial. Here $f$ is either a PL-embedding of the boundary of the standard $(n+1)$-dimensional simplex into $\mathbb{R}^{n+2}$ or a smooth embedding of $S^{n}$ into $\mathbb{R}^{n+2}$ given by a $(n+2)$-dimensional vector of trigonometric polynomials with rational coefficients of the spherical angles.
